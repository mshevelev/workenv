\documentclass[unicode,a4paper,10pt]{article}
\textwidth=16cm
\textheight=25cm
\oddsidemargin=0mm
\topmargin=-10mm

%A Few Useful Packages
\usepackage{marvosym}
\usepackage{fontspec} 					%for loading fonts
\usepackage{xunicode,xltxtra,url,parskip} 	%other packages for formatting
\RequirePackage{color,graphicx}
\usepackage[usenames,dvipsnames]{xcolor}
\usepackage{hyphenat}
\usepackage[utf8]{inputenc}
\usepackage[russian]{babel}
% \usepackage[big]{layaureo} 				%better formatting of the A4 page
% \usepackage{fullpage}
% an alternative to Layaureo can be ** \usepackage{fullpage} **
\usepackage{supertabular} 				%for Grades
\usepackage{titlesec}					%custom \section

%Setup hyperref package, and colours for links
\usepackage{hyperref}
\definecolor{linkcolour}{rgb}{0,0.2,0.6}
\hypersetup{colorlinks,breaklinks,urlcolor=linkcolour, linkcolor=linkcolour}

%FONTS
\defaultfontfeatures{Mapping=tex-text}
%\setmainfont[SmallCapsFont = Fontin SmallCaps]{Fontin}
%\setmainfont[SmallCapsFont = Linux Libertine O C]{Linux Libertine O}
\setmainfont[Ligatures=TeX,Numbers=OldStyle]{LinLibertine_R.otf}

%\setmainfont[SmallCapsFont = FreeSans]{FreeSans}

%CV Sections inspired by: 
%http://stefano.italians.nl/archives/26
\titleformat{\section}{\Large\raggedbottom}{}{0em}{}[\titlerule]
\titlespacing{\section}{0pt}{3pt}{3pt}
%Tweak a bit the top margin
%\addtolength{\voffset}{-1.3cm}

%Italian hyphenation for the word: ''corporations''
\hyphenation{im-pre-se}

%-------------WATERMARK TEST [**not part of a CV**]---------------
\usepackage[absolute]{textpos}

\setlength{\TPHorizModule}{30mm}
\setlength{\TPVertModule}{\TPHorizModule}
\textblockorigin{2mm}{0.65\paperheight}
\setlength{\parindent}{0pt}

%--------------------BEGIN DOCUMENT----------------------
\begin{document}

\pagestyle{empty} % non-numbered pages

\font\fb=''[cmr10]'' %for use with \LaTeX command

%--------------------TITLE-------------
\par{
\begin{flushright}
{\Large Mikhail Shevelev}\\
\medskip	%\bigskip
{\normalsize
\begin{tabular}{rr}
    +1(203)550-33-61\\
    \href{mailto:shevelev.mikhail@gmail.com}{shevelev.mikhail@gmail.com}
\end{tabular}
}
\end{flushright}
\par}

%--------------------SECTIONS-----------------------------------
%%Section: Personal Data
%\section{Личные данные}

%\begin{tabular}{rl}
%    \textsc{Дата рождения:} & 1 января 1900 \\
%    \textsc{phone:}     & +7(909)949-39-17\\
%    \textsc{email:}     & \href{mailto:shevelev.mikhail@gmail.com}{shevelev.mikhail@gmail.com}
%\end{tabular}

%Section: Education
\section{Education}
\begin{tabular}{rl}
2010-2012& \textbf{Higher School of Economics}, Faculty of Economics, \textbf{Master of Arts}\\
& Master's Program <<Finance>>\\% (\textbf{целевая группа Сбербанка})\\
& Specialized in \textit{Options Pricing Models}, \textit{Portfolio Management Theory} \\
%& Тема курсовой работы: <<Option Pricing Models>> \\
%& Тема магистерской работы: <<Сравнение эффективности опционных стратегий хеджирования>> \\
%2008-2009& \textbf{Lomonosov Moscow State University}, Faculty of Mechanics and Mathematics, \textbf{postgraduate student}\\
%& Department of Computational Mathematics \\
%& Specialized in \textit{Distributed computing} and \textit{GRID-systems} \\
2003-2008& \textbf{Lomonosov Moscow State University}, Faculty of Mechanics and Mathematics, \textbf{Specialist}\\
& Department of Computational Mathematics \\
& Specialized in \textit{Distributed computing}, \textit{GRID-systems}, Numerical Methods \\
\end{tabular}

\section{Additional education}
\begin{itemize}
	\item \href{https://smiles.skoltech.ru/mlss2019}{2019 Machine Learning Summer School (MLSS Moscow)}
	\item \href{https://courses.edx.org/dashboard/programs/fa06b9c5-fe2b-41d8-a8c1-1bfd9c0d7b07/}{MitX MicroMasters "Statistics and Data Science"} (completed 4 courses, Capstone Exam in Fall 2023)
\end{itemize}
%Section: Work Experience at the top
\section{Experience}
\begin{tabular}{r p{13cm}}

 \emph{\textsc{June 2014}} & \textbf{\href{https://www.worldquant.com/}{WorldQuant, LLC}, Vice President, UBER Team} \\
\textsc{now}
&\footnotesize{Developed an UBER Trade Module that transforms a Forecast into Strategy by performing multiperiod P\&L-Impact optimization and Partial Risk Factor Neutralization.}\\
&\footnotesize{Developed and operated several Strategies on Equity Markets.}\\
&\footnotesize{Implemented approximation of UBER Impact Model for daily Strategies.}\\
&\footnotesize{Implemented production version of UBER Risk Model.}\\
&\footnotesize{Developed various tools for Portfolio Managers (e.g. UBER Simulator) and Alpha Researchers.}\\
&\footnotesize{Supervised a small team of Researchers on Machine Research.}\\
&\footnotesize{Managed <<MMlib>> and <<CAT>> Projects.}\\

 \emph{\textsc{December 2012}} & \textbf{\href{https://www.worldquant.com/}{WorldQuant Research (Eurasia) LLC}, Quantitative Researcher} \\
\textsc{May 2014}
&\footnotesize{Researched daily and intraday Equity Alphas.}\\
&\footnotesize{Developed <<MMLib>> Expression Library for HUMAN and MACHINE Alpha Research (used by $\sim$20 Researchers).}\\
&\footnotesize{Developed a working prototype of a framework for MACHINE Alpha Research later used by two Teams in Russia.}\\
&\footnotesize{Received 5 monthly and quarterly awards (e.g. Best Alpha Award, Unique Alpha Award, Best Rookie Award).}\\


 \emph{\textsc{October 2011}} & \textbf{\href{http://www.rambler.ru}{Rambler}, Technical Leader of <<User Model>> Group / Researcher} \\
\textsc{December 2012}
&\footnotesize{Conducted research and development of machine learning algorithms in the scope of <<User Model>> project.}\\
&\footnotesize{Provided production quality implementation of machine learning algorithms using Hadoop and Hive.}\\
&\footnotesize{Developed a news-recommendation system at <<\href{http://news.rambler.ru}{Rambler News}>>.}\\
&\footnotesize{Implemented an algorithm for gender classification of Internet users based on anonymized Web browsing history.}\\
&\footnotesize{Performed cluster analysis of Internet users to extract their primary interests.}\\


 \emph{\textsc{May 2009}} & \textbf{\href{http://www.tsys.com}{TSYS Inc}, Senior Analyst of Authorization and Cryptographic systems / Software Developer} \\
\textsc{October 2011}
&\footnotesize{Developed a TCP-server application managing a large network of ATMs.} \\
&\footnotesize{Developed an HSM management module for encryption/decryption of payment card transactions.} \\
%&\footnotesize{Project leader activities during installations/upgrades at \href{http://www.tfskok.pl/}{TF SKOK SA}, 
%				\href{http://privatbank.ua/}{Privat Bank}, \href{http://www.platina.ru/}{Platina Bank}, 
%				on-site installation of NCrypt at \href{http://www.unb.com}{UNB} (Abu Dhabi)}\\

% \emph{\textsc{May 2009}} & \textbf{\href{http://www.tsys.com}{TSYS Inc}, programmer-analyst of authorization and cryptographic systems} \\
%\textsc{February 2011}
%&\footnotesize{Successfully completed the development of stable versions of ATM Controller, NCRYPT and Web ATM Device Monitor products, 
%						participated in on-site installations at Unicredit Bank and Moscow Credit Bank} \\
%&\footnotesize{Successfully supported VISA VSDC and Mastercard MCHIP certifications at Moscow Credit Bank} \\
 
 \emph{\textsc{July 2008}} & \textbf{\href{http://www.imec.msu.ru/}{Institute of Mechanics (Lomonosov Moscow State University)}, Junior Researcher} \\
\textsc{April 2009}
&\footnotesize{Proposed and implemented a framework for development of distributed GRID applications.}\\
&\footnotesize{Implemented distributed algorithms to solve some real-life problems of mechanics and cryptography using the above mentioned framework.}\\
&\footnotesize{Two articles have been published following research.}\\ 
%&\footnotesize{Maintanainance of experimental GRID-system and research in distributed computing} \\
\multicolumn{2}{c}{} 
\end{tabular}

\section{Areas of Expertise}
\textbf{Programming}: distributed computing, fault-tolerant systems, data structures and algorithms,  large scale data processing, object-oriented programming, functional programming\\
\textbf{Mathematics}: numerical methods, statistics, machine learning\\
%\textbf{Payment card industry}: acquiring, issuing, ATM communication protocol, HSMs, cryptography\\
\textbf{Finance}: portfolio management theory, options pricing models \\

%\section{Key Skills}
%\textbf{Distributed Computing}: MapReduce (Hadoop, Hive), Grid-computing (Globus Toolkit), POSIX threads, MPI\\
%\textbf{Programming languages}: Python, С++, Haskell, Oracle PL SQL, bash, sed, awk\\
%\textbf{Mathematical and statistical packages}: R, Octave, Eviews, Stata, MS Excel\\
%\textbf{DBMS}: Redis, Berkley DB, Oracle, MySQL, SQLite\\
%\textbf{OS}: Linux, Windows
%\textbf{Misc. tools}: MS Excel, MS Word, MS Project, MS Visio, JIRA, Google Docs, LaTeX.
 
%%Section: Languages
%\section{Foreign Languages}
%%\begin{tabular}{rl}
%\textsc{English:} fluent
%%\end{tabular}

\section{Publications}
1. E.V.~Shchepin, M.V.~Shevelev, N.E.~Shchepin (2003), "On topology of the number 64",
\textit{Chebyshev collection, v.4, i.4, pp. 153-172}\\
\href{http://www.mi.ras.ru/~scepin/64s.pdf}{http://www.mi.ras.ru/~scepin/64s.pdf}\\
\href{http://tsput.ru/res/math/cheb/attachments/127_chepin.rar}{http://tsput.ru/res/math/cheb/attachments/127\_chepin.rar}

2. Vasenin V.A., Inyukhin A.V., Shevelev M.V. (2009), "Ideas, Solutions, and Current State of the Grid Computing Testbed", 
\textit{application to "Informational technologies" №7/2009}\\
\href{http://novtex.ru/IT/it2009/number07\_pril.htm}{http://novtex.ru/IT/it2009/number07\_pril.htm}

3. Vasenin V.A., Inyukhin A.V., Shevelev M.V. (2010), "A fragment of geographically distributed data-processing environment based on the methodology of GRID", 
\textit{"Informational technologies"\ №1/2010, pp. 63-64}\\
\href{http://elibrary.ru/item.asp?id=13008717}{http://elibrary.ru/item.asp?id=13008717}

\section{Additional Information}
Prizewinner of several conferences and competitions in mathematics and programming:
\begin{itemize}
\item International online contest "ICFP Programming contest 2009"
\item International conference "Third Annual Kolmogorov readings -- 2003", Moscow, Russia
\item International school conference "Third Kharitonov readings -- 2003", Sarov, Russia
\item Regional mathematical olympiad "Lomonosov -- 2003", Moscow
\end{itemize}

%\section{Направления Сбербанка, представляющие наибольший интерес}
%Производные финансовые инструменты, quantitative analyst, управление инвестиционным портфелем, риск-менеджмент.

%\section{Recomendations}
%\begin{tabular}{lll}
%A.N.~Burenin& MGIMO, prof., Head of Stock Market Department& \href{mailto:anburenin@mail.ru}{anburenin@mail.ru} \\
%M.V.~Teslenko& TSYS, head of ATM Controller department & \href{mailto:maxim.teslenko@ctl.com}{maxim.teslenko@ctl.com}
%\end{tabular}

\end{document}
